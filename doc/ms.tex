% mnras_template.tex
%
% LaTeX template for creating an MNRAS paper
%
% v3.0 released 14 May 2015
% (version numbers match those of mnras.cls)
%
% Copyright (C) Royal Astronomical Society 2015
% Authors:
% Keith T. Smith (Royal Astronomical Society)

% Change log
%
% v3.0 May 2015
%    Renamed to match the new package name
%    Version number matches mnras.cls
%    A few minor tweaks to wording
% v1.0 September 2013
%    Beta testing only - never publicly released
%    First version: a simple (ish) template for creating an MNRAS paper

%%%%%%%%%%%%%%%%%%%%%%%%%%%%%%%%%%%%%%%%%%%%%%%%%%
% Basic setup. Most papers should leave these options alone.
\documentclass[a4paper,fleqn,usenatbib]{mnras}

% MNRAS is set in Times font. If you don't have this installed (most LaTeX
% installations will be fine) or prefer the old Computer Modern fonts, comment
% out the following line
\usepackage{newtxtext,newtxmath}
% Depending on your LaTeX fonts installation, you might get better results with one of these:
%\usepackage{mathptmx}
%\usepackage{txfonts}

% Use vector fonts, so it zooms properly in on-screen viewing software
% Don't change these lines unless you know what you are doing
\usepackage[T1]{fontenc}
\usepackage{ae,aecompl}


%%%%% AUTHORS - PLACE YOUR OWN PACKAGES HERE %%%%%

% Only include extra packages if you really need them. Common packages are:
\usepackage{graphicx}	% Including figure files
\usepackage{amsmath}	% Advanced maths commands
\usepackage{amssymb}	% Extra maths symbols
\usepackage{dsfont}
\usepackage{microtype}
\usepackage[utf8]{inputenc}

%%%%%%%%%%%%%%%%%%%%%%%%%%%%%%%%%%%%%%%%%%%%%%%%%%

%%%%% AUTHORS - PLACE YOUR OWN COMMANDS HERE %%%%%

% Please keep new commands to a minimum, and use \newcommand not \def to avoid
% overwriting existing commands. Example:
%\newcommand{\pcm}{\,cm$^{-2}$}	% per cm-squared

%%%%%%%%%%%%%%%%%%%%%%%%%%%%%%%%%%%%%%%%%%%%%%%%%%

\newcommand{\params}{\boldsymbol{\theta}}
\newcommand{\data}{\boldsymbol{D}}

%%%%%%%%%%%%%%%%%%% TITLE PAGE %%%%%%%%%%%%%%%%%%%

% Title of the paper, and the short title which is used in the headers.
% Keep the title short and informative.
\title[]
{Probabilistic and decision theoretic catalogs}
    
\author[Brewer, Malz, and Leung]{%
  Brendon~J.~Brewer$^{1}$\thanks{To whom correspondence should be addressed. Email: {\tt bj.brewer@auckland.ac.nz}}, Alex Malz$^2$, Daisy Leung$^3$\\
  \medskip\\
  $^1$Department of Statistics, The University of Auckland, Private Bag 92019,
        Auckland 1142, New Zealand\\
  $^2$NYU\\
  $^3$Cornell}
% These dates will be filled out by the publisher
\date{}

% Enter the current year, for the copyright statements etc.
\pubyear{2016}

% Don't change these lines
\begin{document}
\label{firstpage}
\pagerange{\pageref{firstpage}--\pageref{lastpage}}
\maketitle

% Abstract of the paper
\begin{abstract}
\end{abstract}

% Select between one and six entries from the list of approved keywords.
% Don't make up new ones.
\begin{keywords}
methods: data analysis --- methods: statistical
\end{keywords}

%%%%%%%%%%%%%%%%%%%%%%%%%%%%%%%%%%%%%%%%%%%%%%%%%%

%%%%%%%%%%%%%%%%% BODY OF PAPER %%%%%%%%%%%%%%%%%%

\section{Introduction}


The implementation is in C++ and makes use of the DNest4 package \citep{dnest4},
which implements the Diffusive Nested Sampling algorithm \citep{dns}.

\section{Decision Theory}
In Bayesian inference, the posterior distribution for the parameters
$\params$ given data $\data$ is proporitional to the prior $p(\params)$
times the likelihood $p(\data | \params)$.
\begin{align}
p(\params | \data) &= \frac{p(\params)p(\data | \params)}{p(\data)}.
\end{align}
Bayesian inference (i.e., probability theory) is used to model uncertainty
because any assignment of real values to
ordered pairs of propositions $A$, $B$, that satisfies basic
symmetries, is equivalent to a conditional probability $P(A | B)$.
\citep{knuth2012foundations}.


Now we have a posterior distribution $p(C|\data)$ for the catalog
\begin{eqnarray}
C = \left\{N,\{x_i, y_i, \{f_{ij}\}\}_{i=1}^N \right\}
\end{eqnarray}
we might want to choose a point estimate catalog $\hat{C}$ which
maximises the expected value of a utility function $U(\hat{C}, C)$.\\

The expected utility is
\begin{eqnarray}
\mathds{E}\left[U(\hat{C}, C)\right]
&=&
\int U(\hat{C}, C) p(C | \data) \, dC,
\end{eqnarray}
which can be approximated by averaging over a set $\{C_i\}_{i=1}^n$
of catalogs produced from a Monte Carlo sampling of the posterior distribution
$p(C|\data)$:
\begin{eqnarray}
\mathds{E}\left[U(\hat{C}, C)\right]
&\approx&
\frac{1}{n} \sum_{i=1}^n U(\hat{C}, C_i).
\end{eqnarray}
Once we have sampled the posterior, we can do an optimisation to choose the
catalog estimate $\hat{C}$ which maximises the approximate expected utility.

\section{A utility aimed at the ``next image''}

I propose
\begin{eqnarray}
U(\hat{C}, C) &=& \textnormal{Div}\left(m(x, y; \hat{C}); m(x, y; C)\right)
\end{eqnarray}
where $\textnormal{Div}$ is unique divergence of
two measures \citep{knuth2012foundations}.

, and the two $m$s are measures over the sky produced from the
point estimate catalog and the true catalog respectively.


\section{A utility function aimed at a ``particular star''}


\section{A utility function aimed at the luminosity function}



\section*{Acknowledgements}
This work was funded by a Marsden Fast Start grant from the Royal Society of
New Zealand, and initiated while on research and study leave supported by
the University of Auckland. We would also like to thank the Astro Hack Week
2015 organizers.

%%%%%%%%%%%%%%%%%%%%%%%%%%%%%%%%%%%%%%%%%%%%%%%%%%

%%%%%%%%%%%%%%%%%%%% REFERENCES %%%%%%%%%%%%%%%%%%

% The best way to enter references is to use BibTeX:

\bibliographystyle{mnras}
\bibliography{references} % if your bibtex file is called example.bib


%%%%%%%%%%%%%%%%%%%%%%%%%%%%%%%%%%%%%%%%%%%%%%%%%%

%%%%%%%%%%%%%%%%% APPENDICES %%%%%%%%%%%%%%%%%%%%%

%\appendix
%\section{Some extra material}

%%%%%%%%%%%%%%%%%%%%%%%%%%%%%%%%%%%%%%%%%%%%%%%%%%


% Don't change these lines
\bsp	% typesetting comment
\label{lastpage}
\end{document}

% End of mnras_template.tex








\documentclass[a4paper, 12pt]{article}
\usepackage{dsfont}
\usepackage[left=2.5cm, right=2.5cm]{geometry}

\newcommand{\data}{\textnormal{data}}

\begin{document}



\end{document}

