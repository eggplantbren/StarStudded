% mnras_template.tex
%
% LaTeX template for creating an MNRAS paper
%
% v3.0 released 14 May 2015
% (version numbers match those of mnras.cls)
%
% Copyright (C) Royal Astronomical Society 2015
% Authors:
% Keith T. Smith (Royal Astronomical Society)

% Change log
%
% v3.0 May 2015
%    Renamed to match the new package name
%    Version number matches mnras.cls
%    A few minor tweaks to wording
% v1.0 September 2013
%    Beta testing only - never publicly released
%    First version: a simple (ish) template for creating an MNRAS paper

%%%%%%%%%%%%%%%%%%%%%%%%%%%%%%%%%%%%%%%%%%%%%%%%%%
% Basic setup. Most papers should leave these options alone.
\documentclass[a4paper,fleqn,usenatbib]{mnras}

% MNRAS is set in Times font. If you don't have this installed (most LaTeX
% installations will be fine) or prefer the old Computer Modern fonts, comment
% out the following line
\usepackage{newtxtext,newtxmath}
% Depending on your LaTeX fonts installation, you might get better results with one of these:
%\usepackage{mathptmx}
%\usepackage{txfonts}

% Use vector fonts, so it zooms properly in on-screen viewing software
% Don't change these lines unless you know what you are doing
\usepackage[T1]{fontenc}
\usepackage{ae,aecompl}


%%%%% AUTHORS - PLACE YOUR OWN PACKAGES HERE %%%%%

% Only include extra packages if you really need them. Common packages are:
\usepackage{graphicx}	% Including figure files
\usepackage{amsmath}	% Advanced maths commands
\usepackage{amssymb}	% Extra maths symbols
\usepackage{dsfont}
\usepackage{microtype}
\usepackage[utf8]{inputenc}

%%%%%%%%%%%%%%%%%%%%%%%%%%%%%%%%%%%%%%%%%%%%%%%%%%

%%%%% AUTHORS - PLACE YOUR OWN COMMANDS HERE %%%%%

% Please keep new commands to a minimum, and use \newcommand not \def to avoid
% overwriting existing commands. Example:
%\newcommand{\pcm}{\,cm$^{-2}$}	% per cm-squared

%%%%%%%%%%%%%%%%%%%%%%%%%%%%%%%%%%%%%%%%%%%%%%%%%%

\newcommand{\params}{\boldsymbol{\theta}}
\newcommand{\data}{\boldsymbol{D}}
\newcommand{\catalog}{\boldsymbol{C}}
\newcommand{\info}{\boldsymbol{I}}

%%%%%%%%%%%%%%%%%%% TITLE PAGE %%%%%%%%%%%%%%%%%%%

% Title of the paper, and the short title which is used in the headers.
% Keep the title short and informative.
\title[]
{Probabilistic and decision theoretic catalogs}
    
\author[Brewer, Malz, Leung, and Lang]{%
  Brendon~J.~Brewer$^{1}$\thanks{To whom correspondence should be addressed. Email: {\tt bj.brewer@auckland.ac.nz}}, Alex Malz$^2$, Daisy Leung$^3$, Dustin Lang$^4$\\
  \medskip\\
  $^1$Department of Statistics, The University of Auckland, Private Bag 92019,
        Auckland 1142, New Zealand\\
  $^2$NYU\\
  $^3$Cornell\\
  $^4$Dunno}
% These dates will be filled out by the publisher
\date{}

% Enter the current year, for the copyright statements etc.
\pubyear{2016}

% Don't change these lines
\begin{document}
\label{firstpage}
\pagerange{\pageref{firstpage}--\pageref{lastpage}}
\maketitle

% Abstract of the paper
\begin{abstract}
Detecting and characterizing the sources in an image is a fundamental task
in astronomy. This can be regarded as an inference problem, where the goal
is to determine what statements about the objects are plausible, given the
data and some assumed prior information. In this paper we present a new
implementation of this idea for the case of crowded stellar fields.
We also formally address the question of how to choose a single optimal
catalog for publication, using concepts from decision theory.
Due to the computational cost, we suggest this method is only suitable for
very small images which are of particular interest.
\end{abstract}

% Select between one and six entries from the list of approved keywords.
% Don't make up new ones.
\begin{keywords}
methods: data analysis --- methods: statistical
\end{keywords}

%%%%%%%%%%%%%%%%%%%%%%%%%%%%%%%%%%%%%%%%%%%%%%%%%%

%%%%%%%%%%%%%%%%% BODY OF PAPER %%%%%%%%%%%%%%%%%%

\section{Introduction}
How many sources are in an image? Where are they located, how bright are they,
and what are their other properties (e.g. shape-related properties)? These
are fundamental questions in astronomy.


\citet{brewer2013probabilistic} addressed this question from the point of view
of Bayesian inference, showing how the posterior distribution over the space of possible catalogs can be computed and can enable greater scientific output from
a dataset, provided the modelling assumptions are sufficiently accurate. In
particular, the frequency distribution of fluxes at the faint end was much
more accurately recovered using inference than using standard object detection
package SExtractor.

In this paper, we describe a new implementation of the
\citet{brewer2013probabilistic}
method, which samples the posterior distribution for possible catalogs
given the data. 

The implementation is in C++ and makes use of the DNest4 package
\citep{dnest4}, which implements the Diffusive Nested Sampling algorithm
\citep{dns}. We also discuss the question of how to choose one optimal catalog
rather than presenting a sample of plausible catalogs. We do this formally
using decision theory.

The methods presented here are very computationally intensive, as is common
for MCMC in thousands of dimensions. Other approximations, such as
`variational Bayes' \citep[as used by e.g.,][]{regier2016learning} may be
required for larger images. Therefore, we suggest this software only be used
on small patches of imaging (less than about 200 $\times$ 200 pixels) that
are of particular interest.


\section{Inference and decision theory}
In Bayesian inference, the posterior distribution for the parameters
$\params$ given data $\data$ and prior information $\info$ is proportional
to the prior $p(\params | \info)$
times the likelihood $p(\data | \params, \info)$.
\begin{align}
p(\params | \data, \info) &=
    \frac{p(\params | \info)p(\data | \params, \info)}{p(\data | \info)}.
\end{align}
Bayesian inference (i.e., probability theory) is used to model uncertainty
because any assignment of real values to
ordered pairs of propositions $(A, B)$, that satisfies basic
symmetries, is equivalent to a conditional probability $P(A | B)$.
\citep{knuth2012foundations}. The posterior distribution quantifies the degree
to which the data $\data$ implies certain conclusions about the value of
$\theta$, within the context of $\info$.

Now we have a posterior distribution $p(C|\data, \info)$ for the catalog
\begin{eqnarray}
C = \left\{N,\{x_i, y_i, \{f_{ij}\}\}_{i=1}^N \right\}
\end{eqnarray}
we might want to choose a point estimate catalog $\hat{C}$ which
maximises the expected value of a utility function $U(\hat{C}, C)$.\\

The expected utility is
\begin{eqnarray}
\mathds{E}\left[U(\hat{C}, C)\right]
&=&
\int U(\hat{C}, C) p(C | \data) \, dC,
\end{eqnarray}
which can be approximated by averaging over a set $\{C_i\}_{i=1}^n$
of catalogs produced from a Monte Carlo sampling of the posterior distribution
$p(C|\data)$:
\begin{eqnarray}
\mathds{E}\left[U(\hat{C}, C)\right]
&\approx&
\frac{1}{n} \sum_{i=1}^n U(\hat{C}, C_i).
\end{eqnarray}
Once we have sampled the posterior, we can do an optimisation to choose the
catalog estimate $\hat{C}$ which maximises the approximate expected utility.

Idea for quick implementation: use the DNest4 samples ({\tt sample.txt})
as the set of possible decisions.

\section{A utility related to images}
Again, let $\hat{C}$ be the 

I propose
\begin{eqnarray}
U(\hat{C}, C) &=& \textnormal{Div}\left(m(x, y; \hat{C}); m(x, y; C)\right)
\end{eqnarray}
where $\textnormal{Div}$ is the unique divergence of
two measures \citep{knuth2012foundations}.

The two $m(x, y)$ functions are measures over the sky produced from the
point estimate catalog and the true catalog respectively. They cannot be
measures where the stars create point masses, because the divergence would
be infinite unless the stars coincided.

The procedure here resembles how we fit the catalog to the images in the
first place. But we're not trying to optimize the fit to the image. Instead,
we're trying to find the posterior sample that is as `close' as possible to
all of the other posterior samples, as defined by the divergence. That catalog
that is most representative of all catalogs that are probable under the
posterior distribution is the best single catalog we could asset.

\section{A utility function aimed at a ``particular star''}


\section{A utility function aimed at the luminosity function}


\section{Compromising between goals}
A utility function that compromises between the above goals.



\section*{Acknowledgements}
This work was funded by a Marsden Fast Start grant from the Royal Society of
New Zealand, and initiated while on research and study leave supported by
the University of Auckland. We would also like to thank the Astro Hack Week
2015 organizers.

%%%%%%%%%%%%%%%%%%%%%%%%%%%%%%%%%%%%%%%%%%%%%%%%%%

%%%%%%%%%%%%%%%%%%%% REFERENCES %%%%%%%%%%%%%%%%%%

% The best way to enter references is to use BibTeX:

\bibliographystyle{mnras}
\bibliography{references} % if your bibtex file is called example.bib


%%%%%%%%%%%%%%%%%%%%%%%%%%%%%%%%%%%%%%%%%%%%%%%%%%

%%%%%%%%%%%%%%%%% APPENDICES %%%%%%%%%%%%%%%%%%%%%

%\appendix
%\section{Some extra material}

%%%%%%%%%%%%%%%%%%%%%%%%%%%%%%%%%%%%%%%%%%%%%%%%%%


% Don't change these lines
\bsp	% typesetting comment
\label{lastpage}
\end{document}

% End of mnras_template.tex


\end{document}

